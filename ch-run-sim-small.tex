\chapter{Running Simulations}
\label{cha:run-sim}

\section{Introduction}
\label{cha:run-sim:intro}

This chapter describes how to run simulations. It covers basic usage, user
interfaces, running simulation campaigns, and many other topics.

\section{Simulation Executables vs Libraries}
\label{sec:run-sim:running}

As we have seen in the \textit{Build} chapter, simulations may be compiled to an
executable or to a shared library. When the build output is an executable,
it can be run directly. For example, the Fifo example simulation can be
run with the following command:

\begin{commandline}
$ ./fifo
\end{commandline}

Simulations compiled to a shared library can be run using the \fprog{opp\_run}
program. For example, if we compiled the Fifo simulation to a
shared library on Linux, the build output would be a \ttt{libfifo.so} file that
could be run with the following command:

\begin{commandline}
$ opp_run -l fifo
\end{commandline}

